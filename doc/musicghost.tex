% vim: ts=2 sts=2 sw=2

\documentclass[a4paper,12pt]{report}
\usepackage[finnish]{babel}
\usepackage[utf8]{inputenc}
%\usepackage[T1]{fontenc}
\usepackage{graphicx}

\title{\includegraphics[width=5em]{logo}\\\vspace{1em}Levytietokanta\\
  \large{Tietokantasovellus, kevät 2011}\\}
\author{Tuomo Lempiäinen\\tuomo.lempiainen@helsinki.fi \and
Hanna Nieminen\\hanna.m.nieminen@helsinki.fi}

\begin{document}

\maketitle

\tableofcontents

\chapter{Määrittely}

\section{Johdanto}

\subsection{Järjestelmän tarkoitus}
Järjestelmän avulla sen ylläpitäjä voi pitää kirjaa levykokoelmastaan.
Järjestelmään on mahdollista luoda useita erillisiä listoja, jolloin on
mahdollista erotella esimerkiksi omistamansa levyt ja tulevat hankinnat
toisistaan.  Lisäksi järjestelmästä voi tehdä julkisen, jolloin myös vierailijat
voivat selata levylistaa ja yksittäisten levyjen tietoja.

\subsection{Toimintaympäristö}
Järjestelmää käytetään WWW-selaimen kautta. Sen ajamiseen tarvitaan
Apache-palvelin varustettuna PHP5-tulkilla sekä PostgreSQL-tietokanta.

\subsection{Rajaukset}
Kaikki ominaisuudet, joita ei ole tässä dokumentissa erikseen mainittu, on
oletusarvoisesti rajattu järjestelmän ulkopuolelle.

\subsection{Toteutusympäristö}
Toteutus pohjautuu Atomik-frameworkin versioon 2.2.2.

\section{Yleiskuva järjestelmästä}

\subsection{Sidosryhmäkaavio}
\vspace{1em}
\includegraphics[width=\linewidth]{sidosryhmakaavio}

\subsection{Käyttäjäryhmät}
\begin{description}
\item[Vierailija] Kuka tahansa sivulla vieraileva käyttäjä. Myös tietokannan
omistaja kuuluu tähän luokkaan.
\item[Ylläpitäjä] Levytietokannan omistaja,
joka ylläpitää tietokantaa.
\end{description}

\section{Käyttötapaukset}

\begin{itemize}

\item Vierailija:
\begin{description}
\item[Levylistan selaaminen] Kuka tahansa voi katsella tietokannan sisältöä.
Listauksessa näkyy kunkin levyn kohdalla sen esittäjä, nimi, julkaisuvuosi ja
formaatti.
\item[Yksittäisen levyn tietojen tutkiminen] Käyttäjä voi valita yksittäisen
levyn, jolloin hänelle näytetään kaikki siihen liitetyt tiedot.
\item[Levyjen hakeminen hakusanalla] Käyttäjä voi antaa hakusanan, jolloin
hänelle näytetään lista niistä levyistä, joiden tiedot sisältävät annetun
hakusanan.
\item[Kirjautuminen sisään] Mikäli vierailija sattuu olemaan tietokannan
omistaja, hän voi kirjautua sisään järjestelmään, jolloin hänestä tulee
ylläpitäjä.
\end{description}

\item Ylläpitäjä:
\begin{description}
\item[Levyn lisääminen] Ylläpitäjä voi lisätä uusia levyjä tietokantaan.
Ylläpitäjä syöttää ainakin levyn tyypin, esittäjän ja nimen sekä valitsee, mille
listalle se lisätään. Vapaaehtoisia tietoja ovat levyn julkaisuvuosi, formaatti,
kansikuva ja vapaamuotoinen kommentti. Esittäjäksi voi valita listasta jonkin jo
tietokannasta löytyvän esittäjän tai vaihtoehtoisesti syöttää kokonaan uuden.
\item[Levyn tietojen muokkaaminen] Ylläpitäjä voi muokata kaikkia yksittäiseen
levyyn liitettyjä tietoja.
\item[Levyn poistaminen] Ylläpitäjä voi poistaa levyn tietokannasta.
\item[Kirjautuminen ulos] Ylläpitäjä voi tarvittavat ylläpitotehtävät tehtyään
kirjautua ulos, jolloin hänestä tulee vierailija.
\end{description}

\end{itemize}

\end{document}
